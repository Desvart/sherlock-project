% This is part of the TFTB Reference Manual.
% Copyright (C) 1996 CNRS (France) and Rice University (US).
% See the file refguide.tex for copying conditions.



\markright{anabpsk}
\section*{\hspace*{-1.6cm} anabpsk}

\vspace*{-.4cm}
\hspace*{-1.6cm}\rule[0in]{16.5cm}{.02cm}
\vspace*{.2cm}



{\bf \large \sf Purpose}\\
\hspace*{1.5cm}
\begin{minipage}[t]{13.5cm}
Binary Phase Shift Keyed (BPSK) signal.
\end{minipage}
\vspace*{.5cm}


{\bf \large \sf Synopsis}\\
\hspace*{1.5cm}
\begin{minipage}[t]{13.5cm}
\begin{verbatim}
[y,am] = anabpsk(N)
[y,am] = anabpsk(N,ncomp)
[y,am] = anabpsk(N,ncomp,f0)
\end{verbatim}
\end{minipage}
\vspace*{.5cm}


{\bf \large \sf Description}\\
\hspace*{1.5cm}
\begin{minipage}[t]{13.5cm}
        {\ty anabpsk} returns a succession of complex sinusoids of {\ty
        ncomp} points each, with a normalized frequency {\ty f0} and an
        amplitude equal to -1 or +1, according to a discrete uniform
        law. Such signal is only 'quasi'-analytic.\\

\hspace*{-.5cm}\begin{tabular*}{14cm}{p{1.5cm} p{8.5cm} c}
Name & Description & Default value\\
\hline
        {\ty N }    & number of points\\
        {\ty ncomp} & number of points of each component & {\ty N/5}\\
        {\ty f0}    & normalized frequency              & {\ty 0.25}\\
  \hline {\ty y}     & signal\\
        {\ty am}    & resulting amplitude modulation     \\
\hline
\end{tabular*}

\end{minipage}
\vspace*{1cm}


{\bf \large \sf Example}
\begin{verbatim}
         [signal,am]=anabpsk(300,30,0.1); 
         subplot(211); plot(real(signal));
         subplot(212); plot(am);
\end{verbatim}
\vspace*{.5cm}


{\bf \large \sf See Also}\\
\hspace*{1.5cm}
\begin{minipage}[t]{13.5cm}
\begin{verbatim}
anafsk, anaqpsk, anaask.
\end{verbatim}
\end{minipage}
\vspace*{.5cm}


{\bf \large \sf Reference}\\
\hspace*{1.5cm}
\begin{minipage}[t]{13.5cm}
[1] W. Gardner {\it Introduction to Random Processes, with Applications to
Signals and Systems}, 2nd Edition, McGraw-Hill, New-York, p. 360 ,1990.  
\end{minipage}
